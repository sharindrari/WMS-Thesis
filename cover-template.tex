% -*-latex-*-
% 
% For questions, comments, concerns or complaints:
% thesis@mit.edu
% 
%
% $Log: cover.tex,v $
% Revision 1.8  2008/05/13 15:02:15  jdreed
% Degree month is June, not May.  Added note about prevdegrees.
% Arthur Smith's title updated
%
% Revision 1.7  2001/02/08 18:53:16  boojum
% changed some \newpages to \cleardoublepages
%
% Revision 1.6  1999/10/21 14:49:31  boojum
% changed comment referring to documentstyle
%
% Revision 1.5  1999/10/21 14:39:04  boojum
% *** empty log message ***
%
% Revision 1.4  1997/04/18  17:54:10  othomas
% added page numbers on abstract and cover, and made 1 abstract
% page the default rather than 2.  (anne hunter tells me this
% is the new institute standard.)
%
% Revision 1.4  1997/04/18  17:54:10  othomas
% added page numbers on abstract and cover, and made 1 abstract
% page the default rather than 2.  (anne hunter tells me this
% is the new institute standard.)
%
% Revision 1.3  93/05/17  17:06:29  starflt
% Added acknowledgements section (suggested by tompalka)
% 
% Revision 1.2  92/04/22  13:13:13  epeisach
% Fixes for 1991 course 6 requirements
% Phrase "and to grant others the right to do so" has been added to 
% permission clause
% Second copy of abstract is not counted as separate pages so numbering works
% out
% 
% Revision 1.1  92/04/22  13:08:20  epeisach

% NOTE:
% These templates make an effort to conform to the MIT Thesis specifications,
% however the specifications can change.  We recommend that you verify the
% layout of your title page with your thesis advisor and/or the MIT 
% Libraries before printing your final copy.
\title{Workflow Management System for Stratosphere:\\Design and Preliminary Implementation}

\author{Suryamita Harindrari}
% If you wish to list your previous degrees on the cover page, use the 
% previous degrees command:
%       \prevdegrees{A.A., Harvard University (1985)}
% You can use the \\ command to list multiple previous degrees
%       \prevdegrees{B.S., University of California (1978) \\
%                    S.M., Massachusetts Institute of Technology (1981)}
\department{IT4BI Consortium and \\Department of Computer Science and Electrical Engineering}

% If the thesis is for two degrees simultaneously, list them both
% separated by \and like this:
% \degree{Doctor of Philosophy \and Master of Science}
\degree{Master of Science in Computer Science and Engineering}

% As of the 2007-08 academic year, valid degree months are September, 
% February, or June.  The default is June.
\degreemonth{August}
\degreeyear{2014}
\thesisdate{August 10, 2014}

%% By default, the thesis will be copyrighted to MIT.  If you need to copyright
%% the thesis to yourself, just specify the `vi' documentclass option.  If for
%% some reason you want to exactly specify the copyright notice text, you can
%% use the \copyrightnoticetext command.  
%\copyrightnoticetext{\copyright IBM, 1990.  Do not open till Xmas.}

% If there is more than one supervisor, use the \supervisor command
% once for each.
\supervisor{Asterios Katsifodimos, PhD}{Thesis Advisor}

% This is the department committee chairman, not the thesis committee
% chairman.  You should replace this with your Department's Committee
% Chairman.
\chairman{Prof. Dr. Volker Markl}{Thesis Supervisor}

% Make the titlepage based on the above information.  If you need
% something special and can't use the standard form, you can specify
% the exact text of the titlepage yourself.  Put it in a titlepage
% environment and leave blank lines where you want vertical space.
% The spaces will be adjusted to fill the entire page.  The dotted
% lines for the signatures are made with the \signature command.
\maketitle

% The abstractpage environment sets up everything on the page except
% the text itself.  The title and other header material are put at the
% top of the page, and the supervisors are listed at the bottom.  A
% new page is begun both before and after.  Of course, an abstract may
% be more than one page itself.  If you need more control over the
% format of the page, you can use the abstract environment, which puts
% the word "Abstract" at the beginning and single spaces its text.

%% You can either \input (*not* \include) your abstract file, or you can put
%% the text of the abstract directly between the \begin{abstractpage} and
%% \end{abstractpage} commands.

% First copy: start a new page, and save the page number.
\cleardoublepage
% Uncomment the next line if you do NOT want a page number on your
% abstract and acknowledgments pages.
% \pagestyle{empty}
\setcounter{savepage}{\thepage}
\begin{abstractpage}
% $Log: abstract.tex,v $
% Revision 1.1  93/05/14  14:56:25  starflt
% Initial revision
% 
% Revision 1.1  90/05/04  10:41:01  lwvanels
% Initial revision
% 
%
%% The text of your abstract and nothing else (other than comments) goes here.
%% It will be single-spaced and the rest of the text that is supposed to go on
%% the abstract page will be generated by the abstractpage environment.  This
%% file should be \input (not \include 'd) from cover.tex.
In this thesis, we design and partly develop a Workflow Management System (WMS) aimed to work on top of Stratosphere, a Big Data platform developed by TU Berlin. The WMS is defined by means of a Domain Specific Language (DSL) written in Scala. Control Flow and data dependencies are automatically detected by static analysis on the Scala code through the following three stages: (1) create a Control Flow Graph as an intermediate representation from Scala AST, (2) detect data dependencies in the graph, and (3) generate code for the underlying system. We cover the implementation of the first stage and provide the algorithm for the subsequent two stages. In the evaluation, we argue over the advantages of this DSL compared to related WMS work in terms of user-friendliness and independence of underlying platform.

\end{abstractpage}

% Additional copy: start a new page, and reset the page number.  This way,
% the second copy of the abstract is not counted as separate pages.
% Uncomment the next 6 lines if you need two copies of the abstract
% page.
% \setcounter{page}{\thesavepage}
% \begin{abstractpage}
% % $Log: abstract.tex,v $
% Revision 1.1  93/05/14  14:56:25  starflt
% Initial revision
% 
% Revision 1.1  90/05/04  10:41:01  lwvanels
% Initial revision
% 
%
%% The text of your abstract and nothing else (other than comments) goes here.
%% It will be single-spaced and the rest of the text that is supposed to go on
%% the abstract page will be generated by the abstractpage environment.  This
%% file should be \input (not \include 'd) from cover.tex.
In this thesis, we design and partly develop a Workflow Management System (WMS) aimed to work on top of Stratosphere, a Big Data platform developed by TU Berlin. The WMS is defined by means of a Domain Specific Language (DSL) written in Scala. Control Flow and data dependencies are automatically detected by static analysis on the Scala code through the following three stages: (1) create a Control Flow Graph as an intermediate representation from Scala AST, (2) detect data dependencies in the graph, and (3) generate code for the underlying system. We cover the implementation of the first stage and provide the algorithm for the subsequent two stages. In the evaluation, we argue over the advantages of this DSL compared to related WMS work in terms of user-friendliness and independence of underlying platform.

% \end{abstractpage}

\cleardoublepage

\section*{Acknowledgments}

I would like to offer my sincere gratitude to my Thesis Advisor, Asterios Katsifodimos, for the opportunity, valuable guidance, honest feedbacks, patience, and motivation given to me throughout the making of this thesis. Most importantly, for believing that I could do the work even though I often times have doubts in myself.

A big thank you goes to Alexander Alexandrov for sharing his great knowledge on compiler and the likes, for showing me how to break down the problem into stages which is a very valuable input for the end product of this thesis, for sharing his technical expertise, and for his valuable feedbacks. 

I would also like to thank Aljoscha Krettek and Andreas Kunft for introducing me to Scala macros and for being helpful whenever I encounter any technical issues along the road. To Andreas, also for giving important feedbacks on my thesis write-up. 

I would like to offer my gratitude to the IT4BI consortium and everyone involved in it as well as the European Commission for the opportunity and for making all this possible especially to Prof. Esteban Zimanyi. Your dream of the IT4BI program opens up ways for many of us and we are forever grateful for that. 

Thank you to my fellow IT4BI classmates and friends, especially my colleagues at TU Berlin - Silvia, Janani, Pino, Faisal, and Prateek. Thank you for the company and great team work during the past two semesters and for your support throughout the thesis development. And to my IT4BI friends in Barcelona, thank you for this two-years of incredible journey.  

I would also like to thank my wonderful sister, friends, and extended family who share their support and motivation during this process. Especially to my sister for always reminding me to stay calm and that I can accomplish my tasks all right. 

I dedicate this thesis to my Mum and my late Dad, the best parents that I could ever hope for, for their everlasting love and support, for showing me the value of hard work, for always reminding me to use give my all to everything that I do, and for always being there even in times when I fail to do so to remind me that in the end, everything is going to be all right. 

%%%%%%%%%%%%%%%%%%%%%%%%%%%%%%%%%%%%%%%%%%%%%%%%%%%%%%%%%%%%%%%%%%%%%%
% -*-latex-*-
