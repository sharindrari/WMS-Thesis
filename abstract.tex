% $Log: abstract.tex,v $
% Revision 1.1  93/05/14  14:56:25  starflt
% Initial revision
% 
% Revision 1.1  90/05/04  10:41:01  lwvanels
% Initial revision
% 
%
%% The text of your abstract and nothing else (other than comments) goes here.
%% It will be single-spaced and the rest of the text that is supposed to go on
%% the abstract page will be generated by the abstractpage environment.  This
%% file should be \input (not \include 'd) from cover.tex.
In this thesis, we design and develop the preliminary scope of a Workflow Management System (WMS) which is aimed to work on top of Stratosphere, one of the emerging large-scale data processing frameworks. The WMS is developed by means of a Domain Specific Language (DSL) which is deeply embedded in Scala high-level programming language. The aim of this workflow DSL is to enable the progammer to define the workflow of complex use cases without having to manually specify the dependencies between the tasks in the workflow. Control Flow and data dependencies are automatically detected by static analysis on the program code using our compiler framework. 

The goal to translate the user program written in our DSL to the target code is achieved through the following three stages: (1) generate a control flow graph as an intermediate representation (IR) from Abstract Syntax Trees (AST) of the program, (2) perform data flow analysis to enrich the graph with data dependencies information, and (3) generate code or job scripts for the underlying system. This research develops the algorithm for each of three stages as well as the implementation of the first stage of the overall process. In the evaluation, we argue over the advantages of this DSL compared to related WMS work in terms of productivity and generality i.e. extensibility to other underlying platforms.
